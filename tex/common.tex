\section{\education}

\cventry{\nov{} 2019 \newline\dec{} 2022
}
{
  \ml%
  {Ph.D in Automatic Control}
  {Doutorado em Controle Automático}
  {Doctorat en Automatique}%
}
{\newline CentraleSupélec/Université Rennes 1%
}
{Rennes}{\france}
{\emph{Sécurité de la Commande Prédictive distribuée sous injection de données faussées}\\
  \textbf{Directeur:} Hervé Guéguen
  \begin{itemize}
    \item Utilisation des méthodes d'estimation/classification pour la détection d'attaques et mitigation de ses effets.
  \end{itemize}
}

\cventry{\sep{} 2017 \newline \sep{} 2018
}
{
  \ml%
  {Master 2 Research in Electronics - Signal, Imaging,Embedded Systems and Control}
  {Master 2 Pesquisa em Eletrônica - Sinal, Imagem,
    Sistemas Embarcados e Controle}
  {Master Recherche Électronique - Signal, Image, Systèmes Embarqués et Automatique}%
}
{\ml%
  {Control Path}
  {Percurso Controle}
  {Parcours Automatique}\newline CentraleSupélec/Université Rennes 1%
}
{Rennes}{\france}{}

\cventry{\sep{} 2016\newline \sep{} 2018
}
{
  \ml%
  {Automatic Systems Engineering - Supélec Formation}
  {Engenharia de Sistemas Automatisados - Formação Supélec}
  {Ingénierie des Systèmes Automatisés - Formation Supélec}}
{\newline CentraleSupélec}
{Rennes}{\france}{}

\cventry{
  \apr{} 2013\newline \aug{} 2019
}
{
  \ml%
  {Control and Automation Engineering Bachelor Degree}
  {Engenharia de Controle e Automação}
  {Ingénierie de Contrôle et d'Automatismes}}
{\newline Universidade Federal do Rio de Janeiro}
{Rio de Janeiro}{\brazil}
{\emph{Identification d'un système mécatronique}\\
  \textbf{Superviseur:} Marcos Vicente de Brito Moreira
  \begin{itemize}
    \item Modélisation en réseau de Petri d'un système mécatronique multi-agent, implementation dans un automate en Ladder, et surveillance pour l'identification global (modèle DAOCT) utilisant python.
  \end{itemize}
}

\cventry{\apr{} 2010\newline \dec{} 2012
}
{
  \ml%
  {Electronics Technical}
  {Técnico em Eletrônica}
  {Technicien en Électronique}}
{\newline Centro Federal de Ensino Tecnológico Celso Suckow da Fonseca}
{\newline Rio de Janeiro}{\brazil}{}

\cventry{
  \ml
  {}
  {abr. 2006\newline dez. 2012\newline}
  {avr. 2006\newline  déc. 2012\newline}
}
{
  \ml
  {Elementary and High School}
  {Ensino Fundamental e Médio}
  {Enseignement secondaire}%
}
{\newline Colégio Pedro II - Unidade Escolar Centro}
{Rio de Janeiro}{\brazil}{}

\section{
  \ml%
  {Experience}
  {Experiência}
  {Expérience Professionnelle}
}
\cventry{\may{} 2023 \newline \may{} 2024\newline}
{
  \ml
  {Post Doctoral Researcher}
  {}
  {Chercheur Postdoctoral}%
}
{\newline Departement: Décision et Optimisation - LAAS/CNRS}
{\newline Toulouse - \france
\newline\ml{Team}{Time}{Équipe}: DIagnostic, Supervision et COnduite - DISCO}
{\newline Localisation relative garantie et scénario d'anticollision de véhicule autonome}
{\textbf{Responsable:} Soheib Fergani
  \begin{itemize}
    \item Plateforme autOCampus: Développement d'algorithmes C++/Matlab pour la localisation des droïdes livreurs sur le campus Université Toulouse III - Paul Sabatier
  \end{itemize}
}

\cventry{\oct{} 2018\newline \feb{} 2019
}
{
  \ml%
  {Engineering Intership}
  {Estágio Engenheiro - Estágio Obrigatório}
  {Stage Ingénieur - Fin d'études}}
{\newline \ml{Team}{Time}{Équipe}: Machine Learning/Fraud Detection - Stone Pagamentos}
{\newline Rio de Janeiro - RJ - Brasil}
{\newline
  \ml%
  {Development of tools used by the internship.\newline Data analyses for payment solutions}
  {Desenvolvimento de programas utilisados pela empresa.\newline Análise
    de dados para soluções de pagamento.}
  {Développement logiciel d'outils utilisés pour l'entreprise. \newline
    Analyse de données pour des solutions de paiement (Monétique)}}
{\ml%
  {
    \begin{itemize}
    \item Programs in Scala using Microsoft SQL Server and other tools
    \item API Rest, Data Streams, State Machines etc using Akka library
    \end{itemize}
  }
  {
    \begin{itemize}
    \item Programas em Scala usando Microsoft SQL Server e outras ferramentas
    \item API Rest, FLuxos de Dados, Máquina de estados etc utilisando
      biblioteca Akka
    \end{itemize}
  }
  {
    \begin{itemize}
    \item Logiciels en Scala, utilisant Microsoft SQL Server, et
      autres outils
    \item API Rest, Flots de données, Machines à états etc utilisant
      bibliothèque Akka
    \end{itemize}
}
}

\cventry{\apr{} 2018 \newline \aug{} 2018
}
{
  \ml%
  {Engineering Internship}
  {Estágio Engenheiro 3º Ano - Fim de Estudos}
  {Stage Ingénieur $\mathbf{3^{ème}} $ Année - Fin d'études}}
{\newline DEA - IRMV - TECH. VEH. INTELLIGENT - Renault}
{\newline Technocentre Renault - Guyancourt - Île de France - France}
{\newline
  \ml%
  {Development of supervision system for autonomous vehicle}
  {Desenvolvimento de sistema de supervisão para veículo autônomo}
  {Développement d'un système de supervision pour véhicule autonome}}
{\ml%
  {
    \begin{itemize}
    \item Interface ROS/Simulink using C++, Python and Matlab/Simulink
    \item State machine using Stateflow
    \end{itemize}
  }
  {
    \begin{itemize}
    \item Interface ROS/Simulink usando C++, Python e Matlab/Simulink
    \item Máquina de estados utilisando Stateflow
    \end{itemize}
  }
  {
    \begin{itemize}
    \item Interface ROS/Simulink utilisant C++, Python et Matlab/Simulink
    \item Machine à états utilisant Stateflow
    \end{itemize}
}
}

\cventry{\nov{} 2017 \newline \apr{} 2018
}
{
  \ml%
  {Industrial Study Project}
  {Projeto de Estudo Industrial}
  {Projet d'Étude Industriel}}
{\newline RTE - Réseau de Transport d'Électricité}
{\newline Rennes $\leftrightarrow$ Paris,\ml{France}{França}{France}}
{\newline
  \ml%
  {Development and validation of automaton to optimize the insertion of Renewable Energies}
  {Desenvolvimento e validação de um autômato para otimizar a inserção
    de Energias Renováveis}
  {Développement et validation d'un automate pour optimiser
    l'insertion des Énergies Renouvelables}}
{
  \begin{itemize}
    \item Étude des différents standards CEI 61131 pour des automates industriels
    \item Rapport sur compatilibité entre cas d'études et les languages utilisés
  \end{itemize}
}


\cventry{
  \jul{} 2017 \newline \aug{} 2017 \newline
}
{
  \ml
  {Engineering Internship}
  {Estágio Engenheiro 2º Ano}
  {Stage Ingénieur $ \mathbf{2^{ème}} $ Année}}
{\newline Institut d'Électronique et de Télécommunication de Rennes}
{\newline Rennes,\ml{France}{França}{France}}
{\newline
  \ml
  {Study of voltage control in Distribution Nets}
  {Estudo de Controle em tensão em redes des distribuição}
  {Étude des régulations en tension des réseaux de distribution}}
{\ml
  {
    \begin{itemize}
    \item Simulation using PowerFactory
    \item Interface between PowerFactory and Simulink
    \item Automation of simulations with Python scripts
    \item Control Validation
    \end{itemize}
  }
  {
    \begin{itemize}
    \item Simulações usando PowerFactory
    \item Interface entre PowerFactory e Simulink
    \item Automação das simulações com scripts em Python
    \item Validação do controle
    \end{itemize}
  }
  {
    \begin{itemize}
    \item Mise en \oe uvre de simulation sur PowerFactory
    \item Interface entre PowerFactory et Simulink
    \item Automatisation des simulations avec scripts en Python
    \item Validation des régulations
    \end{itemize}
  }
}

\cventry{\aug{} 2015 \newline \jun{} 2016
}
{
  \ml
  {Scientific Initiation}
  {Iniciação Científica}
  {Initiation a la recherche scientifique}}
{\newline Laboratório de Processamento de Sinais e Imagens Médicas, UFRJ}
{\newline Rio de Janeiro}
{\brazil}
{
  \ml
  { Design of Secure Control for prosthetic robotic arm for muscle atrophy
    suffering patients
    \begin{itemize}
      \item Modeling and servomotors control
      \item Signal Processing
    \end{itemize}
  }
  {
    Criação de controle de segurança de movimento de braços robóticos prostéticos para
    pacientes com atrofia muscular nos braços
    \begin{itemize}
      \item Modelagem et controle de servomotores
      \item Processamento de Sinais
    \end{itemize}
  }
  { Création de contrôle sécurisé pour le mouvement d'un bras prosthétique
    robotique pour des patients soufrant d'atrophie musculaire
    \begin{itemize}
      \item Modélisation et contrôle de servomoteurs
      \item Traitement des Signaux de différents capteurs (encoders et cellule de charge)
    \end{itemize}
  }
}


\cventry{\apr{} 2013\newline \sep{} 2013
}
{
  \ml
  {}
  {Estágio Técnico em Eletrônica}
  {Stage Technicien en Électronique}}
{\newline Rede Globo - Matriz, TV GLOBO}
{Rio de Janeiro}
{\brazil}
{\ml
  {
    \begin{itemize}
    \item Central de Transmissão e Recepção de Sinais - CTRS
      \begin{itemize}
        \item Transmission et Réception de Signaux Audiovisuels
              \begin{itemize}
                \item Satellites
                \item Fibres Optiques
                \item Internet
              \end{itemize}
      \item Traitement des Signaux (Gamma, Coloration, délai audio etc)
      \end{itemize}
    \end{itemize}
  }
  {
    \begin{itemize}
    \item Central de Transmissão e Recepção de Sinais - CTRS
      \begin{itemize}
      \item Transmissão e Recepção de Sinais Audiovisuais por Satélites,
        \newline Fibra Óptica e Internet
      \item Processamento de Sinais Audiovisuais
      \end{itemize}
    \end{itemize}
  }
  {
    \begin{itemize}
    \item Central de Transmissão e Recepção de Sinais - CTRS
      \begin{itemize}
        \item Transmission et Réception de Signaux Audiovisuels
              \begin{itemize}
                \item Satellites
                \item Fibres Optiques
                \item Internet
              \end{itemize}
      \item Traitement des Signaux (Gamma, Coloration, délai audio etc)
      \end{itemize}
    \end{itemize}
  }
}
\section{
  \ml
  {Teaching}
  {Ensino}
  {Enseignement}
}

\cventry{2023-2024}
{
  \ml
  {Tutoring}
  {Laboratório Automatique 2A}
  {Interventions TP/TD}}
{\newline ENSEEIHT}
{Toulouse}
{\france}
{
  \begin{itemize}
    \item Introduction MATLAB/Simulink 1A (17h30)
    \item Programmation C 1A (17h30)
  \end{itemize}
}

\cventry{2022}
{
  \ml
  {Tutoring}
  {Laboratório Automatique 2A}
  {Interventions TP}}
{\newline ECAM}
{Rennes}
{\france}
{
  \begin{itemize}
    \item Analyse et commande dans l'espace d'état 2A (18h)
    \item Asservissement 2A (30h)
  \end{itemize}
}

\cventry{2020-2022}
{
  \ml
  {Tutoring}
  {Laboratório Automatique 2A}
  {Interventions TP/TD}}
{\newline CentraleSupélec}
{Rennes}
{\france}
{
  \begin{itemize}
    \item Commande Prédictive pour bâtiment intelligent 2A (15h)
    \item Commande Prédictive 3A (24h)
    \item Automatique 2A (24h)
    \item Projet Optimisation pour Microgrid isolé (10h)
  \end{itemize}
}

\cventry{2014-2015
}
{
  \ml
  {Tutoring}
  {Monitoria}
  {Tutorat}}
{\newline Universidade Federal do Rio de Janeiro}
{Rio de Janeiro}
{\brazil}
{
  \ml
  {
    \begin{itemize}
    \item Logic Circuits:
      \begin{itemize}
      \item Algèbre de Boole, Machines de Mealy et de Moore.
      \item Fonctions logiques combinatoires et séquentielles
      \end{itemize}
    \end{itemize}
  }
  {
    \begin{itemize}
    \item Circuitos lógicos (450h):
      \begin{itemize}
        \item Algebra de Boole
        \item Funções lógicas combinatórias e sequenciais
        \item Máquinas de Mealy e de Moore.
      \end{itemize}
    \end{itemize}
  }
  {
    \begin{itemize}
      \item Circuits logiques (450h):
      \begin{itemize}
        \item Algèbre de Boole
        \item Fonctions logiques combinatoires et séquentielles (Machines de Mealy/Moore).
        \item Dessin et implementation avec série 74XX
      \end{itemize}
    \end{itemize}
    \begin{itemize}
      \item Activités:
            \begin{itemize}
              \item TD de solution d'exercices pour éclairer doutes
              \item Correction TPs
              \item Surveillance d'examens
            \end{itemize}
    \end{itemize}
  }
}

\section{
  \ml
  {Publication}
  {Ensino}
  {Publications}
}
\cventry{2022}
{Security of distributed Model Predictive Control under False Data Injection}
{}
{}{}{Doctoral Thesis\\\url{https://theses.hal.science/tel-04003991v1}}

\cventry{2022}
{Expectation-Maximization based defense mechanism for dMPC}
{}
{}{}{9th IFAC Conference on Networked Systems NECSYS 2022\\\url{https://doi.org/10.1016/j.ifacol.2022.07.238}}

\cventry{2021}
{Detection and mitigation of corrupted information in dMPC based on resource allocation}
{}
{}{}{5th Conference on Control and Fault-Tolerant SYSTOL 2021\\\url{https://doi.org/10.1109/SysTol52990.2021.9595927}}

\cventry{2019}
{Identification of a mechatronic system}
{}
{}{}{Bachelor Thesis\\\url{http://repositorio.poli.ufrj.br/monografias/monopoli10029376.pdf}}

\section{
  \ml
  {Programs}
  {Programas}
  {Logiciels}
}
\cventry{2024}
{locafleet}
{}
{}{}{Implémentation d'un filtre d'estimation d'état garanti, basés en méthodes ensemblistes et Zonotopes Contraints.
  Fin principal est la localisation relative d'un flot de véhicules autonomes et l'utilisation de l'estimation pour des méthodes de contrôle sans collision.
  Intégration avec ROS et démonstrateur pour la plateforme autOCampus de l'université UT3.}
\cventry{2021}
{pendulum}
{}
{}{}{Projet de programmation lettrée pour l'enseignement de simulation des systèmes dynamiques et commande. On utilise un système du type «pendule/chariot» et la simulation tourne sur l'invite de commandes. L'utilisateur peut modifier la commande appliquée instantanément sans recompiler le code de la simulation.  \\\url{https://github.com/Accacio/pendulum}}

\cventry{2019}
{DES-tools}
{}
{}{}{Collection d'outils sur l'invite de commandes pour générer semi-automatiquement des figures et tableaux en \LaTeX\ qui répresentent des systèmes à évènements discret (automates et réseaux de Petri)\\\url{https://github.com/Accacio/DES-tools}}

\cventry{2019}
{DAOCT}
{}
{}{}{Outil sur l'invite de commandes pour identifier un modèle DAOCT (Deterministic Automaton With Outputs And Conditional Transitions) des automates pour le diagnostic de failles à partir de fichiers \texttt{.csv} sortant d'un Automate (CLP)\\\url{https://github.com/Accacio/daoct}}

\section{
  \ml
  {IT Competences}
  {Competências Informáticas}
  {Compétences Informatiques}
}

\cvitem{
  \ml
  {Programming}
  {Programação}
  {Code}}
{
  C,
  C++,
  Matlab,
  Scala,
  Python,
  Java,
  LADDER,
  Grafcet,
  Wolfram~Language,
  Fanuc's~TP,
  \LaTeX,
  SQL,
  Emacs~Lisp,
  C\#,
  Assembly,
  etc
}

\cvitem{
  \ml
  {Tools}
  {Ferramentas}
  {Outils}}
{
  Git,
  Bash,
  % Emacs,
  % \LaTeX,
  Tikz,
  Simulink,
  PowerFactory,
  Siemens'~Step7,
  ROS,
  \mbox{ESTEREL}'S~SCADE~Suite/export~to~VxWorks,
  SysML,
  UML,
  HTML,
  % XML,
  Fanuc's~Roboguide,
  SketchUp,
  Trello,
  Asana,
  Jira,
  Confluence,
  Github Projects,
  Blender,
  Godot,
  Gimp,
  % Photoshop
  etc
}


\cvitem{
  \ml
  {Operating Systems}
  {Sistemas Operacionais}
  {Systèmes d'exploitation}}
{
  \ml
  {Linux and Windows}
  {Linux e Windows}
  {Linux et Windows}
}

\section{
  \ml
  {Languages}
  {Línguas}
  {Languages}
}
\newcommand{\razText}{\ml{Functional}{Razoavelmente}{Décemment}}
\newcommand{\bemText}{\ml{Fluent}{Bem}{Bien}}
\newcommand{\poucoText}{\ml{Basic}{Pouco}{Peu}}
\newcommand{\nativoText}{\ml{Native}{Nativo}{Natif}}

\newcommand{\speakingText}{\ml{Speaking}{Fala}{Parle}}
\newcommand{\comprehendText}{\ml{Listening}{Compreende}{Comprend}}
\newcommand{\readingText}{\ml{Reading}{Lê}{Lit}}
\newcommand{\writingText}{\ml{Writing}{Escreve}{Écrit}}

\newlength{\razlength}
\newlength{\bemlength}
\newlength{\poucolength}
\newlength{\nativolength}

\newlength{\speakinglength}
\newlength{\comprehendlength}
\newlength{\readinglength}
\newlength{\writinglength}

\settowidth{\speakinglength}{\speakingText \qquad}
\settowidth{\comprehendlength}{\comprehendText \qquad}
\settowidth{\readinglength}{\readingText \qquad}
\settowidth{\writinglength}{\writingText \qquad}

\settowidth{\razlength}{\razText \qquad}
\settowidth{\bemlength}{\bemText \qquad}
\settowidth{\poucolength}{\poucoText \qquad}
\settowidth{\nativolength}{\nativoText \qquad}

\newlength{\maxlength}

\setlength{\maxlength}{\maxof{\maxof{\maxof{\speakinglength}{\comprehendlength}}{\maxof{\readinglength}{\writinglength}}}{\maxof{\maxof{\razlength}{\bemlength}}{\maxof{\poucolength}{\nativolength}}}}
\settowidth{\maxlength}{\razText \qquad}

\newcommand{\raz}{\parbox{\maxlength}{\razText}}
\newcommand{\bem}{\parbox{\maxlength}{\bemText}}
\newcommand{\pouco}{\parbox{\maxlength}{\poucoText}}
\newcommand{\nativo}{\parbox{\maxlength}{\nativoText}}

\newcommand{\speaking}{\parbox{\maxlength}{\speakingText}}
\newcommand{\comprehend}{\parbox{\maxlength}{\comprehendText}}
\newcommand{\reading}{\parbox{\maxlength}{\readingText}}
\newcommand{\writing}{\parbox{\maxlength}{\writingText}}

\settowidth{\speakinglength}{\speakingText \quad}
\settowidth{\comprehendlength}{\comprehendText \quad}
\settowidth{\readinglength}{\readingText \quad}
\settowidth{\writinglength}{\writingText \quad}

\cvitem{}{\comprehend \speaking \reading \writing}{}

\cvitem{\ml{Portuguese}{Português}{Portugais}}{\nativo \nativo \nativo \nativo}{}

\cvitem{\ml{French}{Francês}{Français}}{\bem \bem \bem \bem}{}

\cvitem{\ml{English}{Inglês}{Anglais}}{\bem \bem \bem \bem}{}

\cvitem{\ml{German}{Alemão}{Allemand}}{\pouco \pouco \pouco \pouco}{}

\section{
  \ml
  {Prizes}
  {Prêmios e distinção}
  {Prix et distinction}
}

\cvitem{
  \ml
  {}
  {Bolsas}
  {Bourse}}
{
  \ml
  {}
  {Bolsa Duplo Diploma BRAFITEC CAPES}
  {Double Diplôme BRAFITEC CAPES} 2016--2018
}

\cvitem{
  \ml
  {3$^{rd}$}
  {3º Lugar}
  {3ème Place}}
{
  \ml
  {Industrial Robotics Olympics - FANUC - France}
  {Olimpíadas de Robótica Industrial - FANUC - França}
  {Olympiades de la Robotique Industrielle - FANUC France} 2017
}

% \cvitem{
%   \ml
%   {Honors}
%   {M. Honrosa}
%   {M. Honorable}}
% {
%   \ml
%   {Brazilian Public Schools Mathematics Olympics - OBMEP  }
%   {Olimpíadas Brasileiras de Matemática para Escolas Públicas - OBMEP}
%   {Olympiades Brésiliennes des Mathematique des Écoles Publiques}}
\section{
  \ml
  {Interests}
  {Interesses}
  {Intérêts}
}

\renewcommand{\listitemsymbol}{-~} % Changes the symbol used for lists
\cvlistdoubleitem
{
  \ml
  {}
  {Controle}
  {Automatique}
}
{
  \ml
  {}
  {Robótica}
  {Robotique}
}


\cvlistdoubleitem{
  \ml
  {Programming}
  {Programação}
  {Programmation}
}
{
  % \ml
  % {}
  % {Criação de Jogos}
  % {Création de Jeux Vidéo}
    \ml
  {Smart Grids}
  {Smart Grids}
  {Smart Grids}
}

\cvlistdoubleitem{
  \ml
  {}
  {Automobilismo}
  {Automobilisme}
}
{
  \ml
  {}
  {Aeronáutica}
  {Aéronautique}
}

\cvlistdoubleitem{
  \ml
  {}
  {Controle de redes de distribuição}
  {Réseaux de Distribution}
}
{
  \ml
  {}
  {Energias Renováveis}
  {Énergies Renouvelables}
}


\cvlistdoubleitem{
  \ml
  {}
  {Processamento de Sinais}
  {Traitement des Signaux}
}
{
  \ml
  {}
  {Prostéticos}
  {Appareils Prothétiques}
}

\cvlistdoubleitem{
  \ml
  {}
  {Acústica}
  {Acoustique}
}
{
  \ml
  {}
  {Eletrônica}
  {Électronique}
}
