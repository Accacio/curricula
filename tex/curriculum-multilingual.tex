\documentclass[11pt,a4paper,sans]{moderncv}

\moderncvstyle{casual}
\moderncvcolor{green}
\usepackage[portuguese,english]{babel}
\usepackage{lipsum}
\usepackage{genealogytree}
\usepackage{xcolor}
\usepackage[utf8]{inputenc}
% https://en.wikibooks.org/wiki/LaTeX/Page_Layout
% \usepackage{showframe}
\usepackage[top=0.5in, bottom=1.in, left=1.25in, right=1.25in]{geometry}
% \usepackage[scale=0.7]{geometry} % Reduce document margins
%\setlength{\hintscolumnwidth}{3cm} % Uncomment to change the width of the dates column
%\setlength{\makecvtitlenamewidth}{10cm} % For the 'classic' style, uncomment to adjust the width of the space allocated to your name

\addtolength{\footskip}{20pt}

\newif\ifportuguese
\newif\iffrench
\newif\ifenglish

% TODO(accacio): improve this logic
\newcommand{\portuguese}{\portuguesetrue \frenchfalse \englishfalse}
\newcommand{\french}{\frenchtrue \portuguesefalse \englishfalse}
\newcommand{\english}{\englishtrue \frenchfalse \portuguesefalse}

% \portuguese
% \english
\french

\newcommand\ml[3]{\ifenglish    {#1}\fi
  \ifportuguese {#2}\fi
  \iffrench     {#3}\fi}
% \ml{English}{Portuguese}{French}

\newcommand{\now}{%
  \ml
  {(ongoing)}
  {(em curso)}
  {(en cours)}%
}

\firstname{\huge RAFAEL ACCÁCIO} % Your first name
\familyname{NOGUEIRA} % Your last name


% \title{Curriculum Vitæ}

\mobile{+33~$\cdot$~07~$\cdot$~4900~$\cdot$~4237}
\homepage{accacio.gitlab.io}{accacio.gitlab.io}
\email{rafael.accacio.nogueira@gmail.com}
\extrainfo{
  \includegraphics[width=10pt]{pictures/GitHub-Mark-32px} \httplink[Accacio]{github.com/Accacio}
  \footersymbol
  \includegraphics[width=10pt]{pictures/ORCID-icon} \httplink[0000-0001-9341-1877]{orcid.org/0000-0001-9341-1877}
  \footersymbol
  \includegraphics[width=10pt]{pictures/Google-Scholar-logo}
  \httplink[ThbzClMAAAAJ]{scholar.google.com/citations?user=ThbzClMAAAAJ&hl=en}
}

% \photo[70pt][0pt]{pictures/foto_4x5_2}

\begin{document}

\makecvtitle%
\newcommand{\feb}{\ml{feb}{fev.}{fev.}}
\newcommand{\apr}{\ml{apr}{abr.}{avr.}}
\newcommand{\jun}{\ml{jun}{jun.}{juin}}
\newcommand{\jul}{\ml{july}{jul.}{juil.}}
\newcommand{\aug}{\ml{aug}{ago.}{août}}
\newcommand{\sep}{\ml{sep}{set.}{sept.}}
\newcommand{\oct}{\ml{oct}{out.}{oct.}}
\newcommand{\nov}{\ml{nov}{nov.}{nov.}}
\newcommand{\dec}{\ml{dec}{dez.}{déc.}}
\newcommand{\france}{\ml{France}{França}{France}}
\newcommand{\brazil}{\ml{Brazil}{Brasil}{Brésil}}

\section{\ml{Education}{Educação}{Études}}

\cventry{\nov{} 2019 \newline\dec{} 2022
}
{
  \ml%
  {Ph.D in Automatic Control}
  {Doutorado em Controle Automático}
  {Doctorat en Automatique}%
  \ \now%
}
{\newline CentraleSupélec/Université Rennes 1%
}
{Rennes}{\france}{}

\cventry{\sep{} 2017 \newline \sep{} 2018
}
{
  \ml%
  {Master 2 Research in Electronics - Signal, Imaging,Embedded Systems and Control}
  {Master 2 Pesquisa em Eletrônica - Sinal, Imagem,
    Sistemas Embarcados e Controle}
  {Master Recherche Électronique - Signal, Image, Systèmes Embarqués et Automatique}%
}
{\ml%
  {Control Path}
  {Percurso Controle}
  {Parcours Automatique}\newline CentraleSupélec/Université Rennes 1%
}
{Rennes}{\france}{}

\cventry{\sep{} 2016\newline \sep{} 2018
}
{
  \ml%
  {Automatic Systems Engineering - Supélec Formation}
  {Engenharia de Sistemas Automatisados - Formação Supélec}
  {Ingénierie des Systèmes Automatisés - Formation Supélec}}
{\newline CentraleSupélec}
{Rennes}{\france}{}

\cventry{
  \apr{} 2013\newline \dec{} 2019
}
{
  \ml%
  {Control and Automation Engineering Bachelor Degree}
  {Engenharia de Controle e Automação}
  {Ingénierie de Contrôle et d'Automatismes}}
{\newline Universidade Federal do Rio de Janeiro}
{Rio de Janeiro}{\brazil}{}

\cventry{\apr{} 2010\newline \dec{} 2012
}
{
  \ml%
  {Electronics Technical}
  {Técnico em Eletrônica}
  {Technicien en Électronique}}
{\newline Centro Federal de Ensino Tecnológico Celso Suckow da Fonseca}
{\newline Rio de Janeiro}{\brazil}{}

\cventry{
  \ml
  {}
  {abr. 2006\newline dez. 2012\newline}
  {avr. 2006\newline  déc. 2012\newline}
}
{
  \ml
  {Elementary and High School}
  {Ensino Fundamental e Médio}
  {Enseignement secondaire}}
{\newline Colégio Pedro II - Unidade Escolar Centro}
{Rio de Janeiro}{\brazil}{}

\section{
  \ml%
  {Experience}
  {Experiência}
  {Expérience Professionnelle}
}

\cventry{\oct{} 2018\newline \feb{} 2019
}
{
  \ml%
  {Engineering Intership}
  {Estágio Engenheiro - Estágio Obrigatório}
  {Stage Ingénieur - Fin d'études}}
{\newline \ml{Team}{Time}{Équipe}: Machine Learning/Fraud Detection - Stone Pagamentos}
{\newline Rio de Janeiro - RJ - Brasil}
{\newline
  \ml%
  {Development of tools used by the internship.\newline Data analyses for payment solutions}
  {Desenvolvimento de programas utilisados pela empresa.\newline Análise
    de dados para soluções de pagamento.}
  {Développement logiciel d'outils utilisés pour l'entreprise. \newline
    Analyse de données pour des solutions de paiement (Monétique)}}
{\ml%
  {
    \begin{itemize}
    \item Programs in Scala using Microsoft SQL Server and other tools
    \item API Rest, Data Streams, State Machines etc using Akka library
    \end{itemize}
  }
  {
    \begin{itemize}
    \item Programas em Scala usando Microsoft SQL Server e outras ferramentas
    \item API Rest, FLuxos de Dados, Máquina de estados etc utilisando
      biblioteca Akka
    \end{itemize}
  }
  {
    \begin{itemize}
    \item Logiciels en Scala, utilisant Microsoft SQL Server, et
      autres outils
    \item API Rest, Flots de données, Machines à états etc utilisant
      bibliothèque Akka
    \end{itemize}
}
}

\cventry{\apr{} 2018 \newline \aug{} 2018
}
{
  \ml%
  {Engineering Internship}
  {Estágio Engenheiro 3º Ano - Fim de Estudos}
  {Stage Ingénieur $\mathbf{3^{ème}} $ Année - Fin d'études}}
{\newline DEA - IRMV - TECH. VEH. INTELLIGENT - Renault}
{\newline Technocentre Renault - Guyancourt - Île de France - France}
{\newline
  \ml%
  {Development of supervision system for autonomous vehicle}
  {Desenvolvimento de sistema de supervisão para veículo autônomo}
  {Développement d'un système de supervision pour véhicule autonome}}
{\ml%
  {
    \begin{itemize}
    \item Interface ROS/Simulink using C++, Python and Matlab/Simulink
    \item State machine using Stateflow
    \end{itemize}
  }
  {
    \begin{itemize}
    \item Interface ROS/Simulink usando C++, Python e Matlab/Simulink
    \item Máquina de estados utilisando Stateflow
    \end{itemize}
  }
  {
    \begin{itemize}
    \item Interface ROS/Simulink utilisant C++, Python et Matlab/Simulink
    \item Machine à états utilisant Stateflow
    \end{itemize}
}
}

\cventry{\nov{} 2017 \newline \apr{} 2018
}
{
  \ml%
  {Industrial Study Project}
  {Projeto de Estudo Industrial}
  {Projet d'Étude Industriel}}
{\newline RTE - Réseau de Transport d'Électricité}
{\newline Rennes $\leftrightarrow$ Paris,\ml{France}{França}{France}}
{\newline
  \ml%
  {Development and validation of automaton to optimize the insertion of Renewable Energies}
  {Desenvolvimento e validação de um autômato para otimizar a inserção
    de Energias Renováveis}
  {Développement et validation d'un automate pour optimiser
    l'insertion des Énergies Renouvelables}}
{}


\cventry{
  \jul{} 2017 \newline \aug{} 2017 \newline
}
{
  \ml
  {Engineering Internship}
  {Estágio Engenheiro 2º Ano}
  {Stage Ingénieur $ \mathbf{2^{ème}} $ Année}}
{\newline Institut d'Électronique et de Télécommunication de Rennes}
{\newline Rennes,\ml{France}{França}{France}}
{\newline
  \ml
  {Study of voltage control in Distribution Nets}
  {Estudo de Controle em tensão em redes des distribuição}
  {Étude des régulations en tension des réseaux de distribution}}
{\ml
  {
    \begin{itemize}
    \item Simulation using PowerFactory
    \item Interface between PowerFactory and Simulink
    \item Automation of simulations with Python scripts
    \item Control Validation
    \end{itemize}
  }
  {
    \begin{itemize}
    \item Simulações usando PowerFactory
    \item Interface entre PowerFactory e Simulink
    \item Automação das simulações com scripts em Python
    \item Validação do controle
    \end{itemize}
  }
  {
    \begin{itemize}
    \item Mise en \oe uvre de simulation sur PowerFactory
    \item Interface entre PowerFactory et Simulink
    \item Automatisation des simulations avec scripts en Python
    \item Validation des régulations
    \end{itemize}
  }
}

\cventry{\aug{} 2015 \newline \jun{} 2016
}
{
  \ml
  {Scientific Initiation}
  {Iniciação Científica}
  {Initiation a la recherche scientifique}}
{\newline Laboratório de Processamento de Sinais e Imagens Médicas, UFRJ}
{\newline Rio de Janeiro}
{\brazil}
{
  \ml
  { Design of Secure Control for prosthetic robotic arm for muscle atrophy
    suffering patients
    \begin{itemize}
      \item Modeling and servomotors control
      \item Signal Processing
    \end{itemize}
  }
  {
    Criação de controle de segurança de movimento de braços robóticos prostéticos para
    pacientes com atrofia muscular nos braços
    \begin{itemize}
      \item Modelagem et controle de servomotores
      \item Processamento de Sinais
    \end{itemize}
  }
  { Création de contrôle sécurisé pour le mouvement d'un bras prosthétique
    robotique pour des patients soufrant d'atrophie musculaire
    \begin{itemize}
      \item Modelisation et contrôle de servomoteurs
      \item Traitement des Signaux
    \end{itemize}
  }
}


\cventry{\apr{} 2013\newline \sep{} 2013
}
{
  \ml
  {}
  {Estágio Técnico em Eletrônica}
  {Stage Technicien en Électronique}}
{\newline Rede Globo - Matriz, TV GLOBO}
{Rio de Janeiro}
{\brazil}
{\ml
  {
    \begin{itemize}
    \item Central de Transmissão e Recepção de Sinais - CTRS
      \begin{itemize}
      \item Transmission et Réception de Signaux Audiovisuels par Satellites,
        \newline Fibres Optiques et Internet
      \item Traitement des Signaux
      \end{itemize}
    \end{itemize}
  }
  {
    \begin{itemize}
    \item Central de Transmissão e Recepção de Sinais - CTRS
      \begin{itemize}
      \item Transmissão e Recepção de Sinais Audiovisuais por Satélites,
        \newline Fibra Óptica e Internet
      \item Processamento de Sinais Audiovisuais
      \end{itemize}
    \end{itemize}
  }
  {
    \begin{itemize}
    \item Central~de~Transmissão~e~Recepção~de~Sinais - CTRS
      \newline {\em Centre de Transmission e Réception de Signals}
      \begin{itemize}
      \item Transmission et Réception de Signaux Audiovisuels par Satellites,
        \newline Fibres Optiques et Internet
      \item Traitement des Signaux
      \end{itemize}
    \end{itemize}
  }
}
\section{
  \ml
  {Teaching}
  {Ensino}
  {Enseignement}
}


\cventry{2022}
{
  \ml
  {Tutoring}
  {Laboratório Automatique 2A}
  {TP Analyse et commande dans l'espace d'état 2A} (30h)}
{\newline ECAM}
{Rennes}
{\france}
{}

\cventry{2022}
{
  \ml
  {Tutoring}
  {Laboratório Automatique 2A}
  {TP Commande Prédictive pour bâtiment intelligent 2A}  (15h)}
{\newline CentraleSupélec}
{Rennes}
{\france}
{}

\cventry{2022}
{
  \ml
  {Tutoring}
  {Laboratório Automatique 2A}
  {TP Asservissements 2A} (18h)}
{\newline ECAM}
{Rennes}
{\france}
{}

\cventry{2022}
{
  \ml
  {Tutoring}
  {Laboratório Automatique 2A}
  {TP Commande Prédictive 3A} (12h)}
{\newline CentraleSupélec}
{Rennes}
{\france}
{}

\cventry{2021}
{
  \ml
  {Tutoring}
  {Laboratório Automatique 2A}
  {TP Commande Prédictive 3A} (12h)}
{\newline CentraleSupélec}
{Rennes}
{\france}
{}

\cventry{2021}
{\ml
  {Tutoring}
  {Laboratório Automatique 2A}
  {TP Automatique 2A}
(6h)}
{\newline CentraleSupélec}{Rennes}{\france}{}

\cventry{2021}
{
  \ml
  {Tutoring}
  {Laboratório Automatique 2A}
  {Projet Optimisation pour Microgrid isolé} (10h)}
{\newline CentraleSupélec}
{Rennes}
{\france}
{}

\cventry{2020}
{
  \ml
  {Tutoring}
  {Laboratório Automatique 2A}
  {TP Automatique 2A} (18h)}
{\newline CentraleSupélec}
{Rennes}
{\france}
{}

\cventry{\jul{} 2014\newline \jul{} 2015
}
{
  \ml
  {Tutoring}
  {Monitoria}
  {Tutorat} (450h)}
{\newline Universidade Federal do Rio de Janeiro}
{Rio de Janeiro}
{\brazil}
{
  \ml
  {
    \begin{itemize}
    \item Logic Circuits:
      \begin{itemize}
      \item Algèbre de Boole, Machines de Mealy et de Moore.
      \item Fonctions logiques combinatoires et séquentielles
      \end{itemize}
    \end{itemize}
  }
  {
    \begin{itemize}
    \item Circuitos lógicos:
      \begin{itemize}
      \item Algebra de Boole
      \item Funções lógicas combinatórias e sequenciais
      \item Máquinas de Mealy e de Moore.
      \end{itemize}
    \end{itemize}
  }
  {
    \begin{itemize}
    \item Circuits logiques:
      \begin{itemize}
      \item Algèbre de Boole, Machines de Mealy et de Moore.
      \item Fonctions logiques combinatoires et séquentielles
      \end{itemize}
    \end{itemize}
  }
}

\newpage
\section{
  \ml
  {Publication}
  {Ensino}
  {Publications}
}

\cventry{2022}
{Expectation-Maximization based defense mechanism for dMPC}
{}
{}{}{9th IFAC Conference on Networked Systems NECSYS 2022\\\url{https://doi.org/10.1016/j.ifacol.2022.07.238}}

\cventry{2021}
{Detection and mitigation of corrupted information in dMPC based on resource allocation}
{}
{}{}{5th Conference on Control and Fault-Tolerant SYSTOL 2021\\\url{https://doi.org/10.1109/SysTol52990.2021.9595927}}

\cventry{2019}
{Identification of a mechatronic system}
{}
{}{}{Bachelor Thesis\\\url{http://repositorio.poli.ufrj.br/monografias/monopoli10029376.pdf}}


\section{
  \ml
  {IT Competences}
  {Competências Informáticas}
  {Compétences Informatiques}
}

\cvitem{
  \ml
  {Programming}
  {Programação}
  {Code}}
{
  C,
  C++,
  Matlab,
  Scala,
  Python,
  Java,
  LADDER,
  Grafcet,
  Wolfram~Language,
  Fanuc's~TP,
  \LaTeX,
  SQL,
  Emacs~Lisp,
  C\#,
  Assembly,
  etc
}

\cvitem{
  \ml
  {Tools}
  {Ferramentas}
  {Outils}}
{
  Git,
  Bash,
  % Emacs,
  % \LaTeX,
  ESTEREL'S~SCADE~Suite~/~export~to~VxWorks,
  SysML,
  UML,
  Tikz,
  Simulink,
  PowerFactory,
  Siemens'~Step7,
  ROS,
  HTML,
  % XML,
  Fanuc's~Roboguide,
  % SketchUp,
  Trello,
  Asana,
  Jira,
  Confluence,
  Github Projects
  % Gimp,
  % Photoshop
  etc
}


\cvitem{
  \ml
  {Operating Systems}
  {Sistemas Operacionais}
  {Systèmes d'exploitation}}
{
  \ml
  {Linux and Windows}
  {Linux e Windows}
  {Linux et Windows}
}

\section{
  \ml
  {Languages}
  {Línguas}
  {Languages}
}
\newcommand{\razText}{\ml{Functional}{Razoavelmente}{Décemment}}
\newcommand{\bemText}{\ml{Fluent}{Bem}{Bien}}
\newcommand{\poucoText}{\ml{Basic}{Pouco}{Peu}}
\newcommand{\nativoText}{\ml{Native}{Nativo}{Natif}}

\newcommand{\speakingText}{\ml{Speaking}{Fala}{Parle}}
\newcommand{\comprehendText}{\ml{Listening}{Compreende}{Comprend}}
\newcommand{\readingText}{\ml{Reading}{Lê}{Lit}}
\newcommand{\writingText}{\ml{Writing}{Escreve}{Écrit}}

\newlength{\razlength}
\newlength{\bemlength}
\newlength{\poucolength}
\newlength{\nativolength}

\newlength{\speakinglength}
\newlength{\comprehendlength}
\newlength{\readinglength}
\newlength{\writinglength}

\settowidth{\speakinglength}{\speakingText \qquad}
\settowidth{\comprehendlength}{\comprehendText \qquad}
\settowidth{\readinglength}{\readingText \qquad}
\settowidth{\writinglength}{\writingText \qquad}

\settowidth{\razlength}{\razText \qquad}
\settowidth{\bemlength}{\bemText \qquad}
\settowidth{\poucolength}{\poucoText \qquad}
\settowidth{\nativolength}{\nativoText \qquad}

\newlength{\maxlength}

\setlength{\maxlength}{\maxof{\maxof{\maxof{\speakinglength}{\comprehendlength}}{\maxof{\readinglength}{\writinglength}}}{\maxof{\maxof{\razlength}{\bemlength}}{\maxof{\poucolength}{\nativolength}}}}
\settowidth{\maxlength}{\razText \qquad}

\newcommand{\raz}{\parbox{\maxlength}{\razText}}
\newcommand{\bem}{\parbox{\maxlength}{\bemText}}
\newcommand{\pouco}{\parbox{\maxlength}{\poucoText}}
\newcommand{\nativo}{\parbox{\maxlength}{\nativoText}}

\newcommand{\speaking}{\parbox{\maxlength}{\speakingText}}
\newcommand{\comprehend}{\parbox{\maxlength}{\comprehendText}}
\newcommand{\reading}{\parbox{\maxlength}{\readingText}}
\newcommand{\writing}{\parbox{\maxlength}{\writingText}}

\settowidth{\speakinglength}{\speakingText \quad}
\settowidth{\comprehendlength}{\comprehendText \quad}
\settowidth{\readinglength}{\readingText \quad}
\settowidth{\writinglength}{\writingText \quad}

\cvitem{}{\comprehend \speaking \reading \writing}{}

\cvitem{\ml{Portuguese}{Português}{Portugais}}{\nativo \nativo \nativo \nativo}{}

\cvitem{\ml{French}{Francês}{Français}}{\bem \bem \bem \bem}{}

\cvitem{\ml{German}{Alemão}{Allemand}}{\pouco \pouco \pouco \pouco}{}

\cvitem{\ml{English}{Inglês}{Anglais}}{\bem \bem \bem \bem}{}

\section{
  \ml
  {Prizes}
  {Prêmios e distinção}
  {Prix et distinction}
}

\cvitem{
  \ml
  {}
  {Bolsas}
  {Bourse}}
{
  \ml
  {}
  {Bolsa Duplo Diploma BRAFITEC CAPES}
  {Double Diplôme BRAFITEC CAPES}}

\cvitem{
  \ml
  {3$^{rd}$}
  {3º Lugar}
  {3ème Place}}
{
  \ml
  {Industrial Robotics Olympics - FANUC - France}
  {Olimpíadas de Robótica Industrial - FANUC - França}
  {Olympiades de la Robotique Industrielle - FANUC France}}

% \cvitem{
%   \ml
%   {Honors}
%   {M. Honrosa}
%   {M. Honorable}}
% {
%   \ml
%   {Brazilian Public Schools Mathematics Olympics - OBMEP  }
%   {Olimpíadas Brasileiras de Matemática para Escolas Públicas - OBMEP}
%   {Olympiades Brésiliennes des Mathematique des Écoles Publiques}}

\section{
  \ml
  {Interests}
  {Interesses}
  {Intérêts}
}

\renewcommand{\listitemsymbol}{-~} % Changes the symbol used for lists
\cvlistdoubleitem
{
  \ml
  {}
  {Controle}
  {Automatique}
}
{
  \ml
  {}
  {Robótica}
  {Robotique}
}


\cvlistdoubleitem{
  \ml
  {Programming}
  {Programação}
  {Programmation}
}
{
  % \ml
  % {}
  % {Criação de Jogos}
  % {Création de Jeux Vidéo}
    \ml
  {Smart Grids}
  {Smart Grids}
  {Smart Grids}
}

\cvlistdoubleitem{
  \ml
  {}
  {Automobilismo}
  {Automobilisme}
}
{
  \ml
  {}
  {Aeronáutica}
  {Aéronautique}
}

\cvlistdoubleitem{
  \ml
  {}
  {Controle de redes de distribuição}
  {Régulation de Réseaux de Distribution}
}
{
  \ml
  {}
  {Energias Renováveis}
  {Énergies Renouvelables}
}


\cvlistdoubleitem{
  \ml
  {}
  {Processamento de Sinais}
  {Traitement des Signaux}
}
{
  \ml
  {}
  {Prostéticos}
  {Appareils Prothétiques}
}

\cvlistdoubleitem{
  \ml
  {}
  {Acústica}
  {Acoustique}
}
{
  \ml
  {}
  {Eletrônica}
  {Électronique}
}

\end{document}
